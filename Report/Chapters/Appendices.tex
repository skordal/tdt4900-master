\begin{appendix}

\chapter{Detailed DMA Description}
\label{app:DMA-arch}
This appendix contains additional details about the DMA implementation, and serves to give a deeper understanding of how each component in the DMA works in details.
Section 1 describes every component in detail, while chapter 2 describes tests used to verify the DMA module.

\section{DMA Architecture}
This section aims to give the reader an insight into every detail of the DMA Module.
Every component is detailed with all inputs, outputs and their functionality, as well as how they interact with each other.

\subsection{Overview}
Figure \ref{fig:TopViewFinalSimple2IO} shows the entire DMA module, its components, and its inputs and outputs.

\begin{figure}[htb]
    \centering
    \includegraphics[width=1.0\textwidth]{Figures/DMA/TopViewFinalSimple2IO}
    \caption{Overview of the DMA module, with its inputs and outputs}
    \label{fig:TopViewFinalSimple2IO}
\end{figure}

\subsubsection{Components}
The DMA module consists of the following components:

\begin{description}
    \item[DMA Controller -] 
    The controlling unit responsible for handling transfer requests, monitor channel activity and sending out interrupts.
    \item[Channels -] 
    A channel is capable of performing a requested transfer, by issuing loads and stores.
    The system currently implements two channels.
    Each channel is split into two sub-modules, called the ``load channel'' and ``store channel''.
    \item[Arbiter -]
    Used to select the next output: an interrupt, a store or a load command.     
    \item[FIFO Buffers -]
    Two FIFO buffers are used to store transfer requests and data.
    \item[Comparator -]
    Used to identify which channel has requested the next data.
    \item[Adapter -]
    Used between the request FIFO buffer and the DMA controller for proper synchronization between their signals.  
\end{description}

\subsubsection{Inputs and outputs}
Inputs and outputs for the DMA toplevel are mainly composed of the the input and outputs of the FIFO buffers and the arbiter.
\begin{description}
    \item[Request FIFO buffer -]
    The inputs are control signals for storing data in the FIFO buffer, and details on 96 bits: 32 for the source address, 32 for the destination address, and 32 for transfer details, including transfer ID from the requesting peripheral and the amount of data to transfer. 
    The output signal is the FIFO full signal, signaling that the buffer is full.
    \item[Data FIFO buffer -] 
    The inputs are a control signal for storing, and 64 bits of details about the data: 32 bits for source address, and 32 bits for the data itself.
    The output signal is a FIFO full signal, signaling that the buffer is full.
    \item[Arbiter -]
    The outputs that passes through the arbiter are a control signal to be used externally, and details on 66 bits: the 2 first bits are used to identify the type of command, the rest are the details (interrupt ID, or load/store address and data).
    The input block signal may be used by the external system to stop the DMA module from sending out any more data if the external system is saturated and must process the current output before receiving any more. 
    The impact of this depends on the external system.
\end{description}

\subsection{DMA Controller State Machine}

The DMA controller is implemented as a state machine.
It is responsible for handling requests to do data transfers, monitoring active channels, and sending out an interrupt signal when a channel has completed its data transfer.
The state machine can be seen in figure \ref{fig:DMAControllerStateMachine}.

\begin{figure}[htb]
    \centering
    \includegraphics[width=1\textwidth]{Figures/DMA/StateMachineFinal}
    \caption{Detailed DMA Controller State Machine}
    \label{fig:DMAControllerStateMachine}
\end{figure}

\subsubsection{States}
The states used in the state machine are IDLE, PROCESSING, WAIT AND INTERRUPT.

\begin{description}
    \item[IDLE -] 
    Entered when there are no requests for data transfer.
    The DMA controller waits until a request signal is received.
    \item[PROCESSING -] 
    Entered when the controller has received a request and there is a free channel to execute the request.
    Outputs from the DMA controller to a free channel is transferred, activating the channel.
    \item[WAIT -]
    Entered when both channels are active and there are no more room for new requests to be processed or when there are no more requests to process while there are still channels active.
    \item [INTERRUPT -]
    Entered when a channel is finished with execution and no longer active; causes the DMA controller to request the arbiter to let it send out an interrupt signal.
    Until the arbiter acknowledges the request, the controller will remain in this state, and no new transfer requests will be processed until done.
    When acknowledged, the controller will send out an interrupt signal with necessary details, including the interrupt ID.
\end{description}

\subsubsection{Inputs}
The DMA controller have the following inputs:

\begin{description}
    \item[clk -]
    Input clock signal.
    \item[reset -]
    Reset signal that reverts state machine back to idle.
    \item[req -]
    Input request signal for transferring data.
    \item[loadDetails -]
    The initial 32-bit source address
    \item[storeDetails -]
    The initial 32-bit destination address
    \item[reqDetails -]
    32-bit details relevant for the transfer.
    Contains the count (bits 32-21), which is used for both offset and counting down the execution to finish, and ID of the requestor (bits 20-13), which is used in the interrupt signal.
    Bits 12--0 are reserved for future use.
    \item[activeCh0 -]
    Monitoring signal from channel 0, signaling that it is currently active.
    \item[activeCh1 -]
    Same as activeCh0, but for channel 1.
    \item[interruptAck -]
    Signal from arbiter that the DMA controller has access to send out interrupt signal from the system.
\end{description}

\subsubsection{Outputs}
These are the following output signals from the DMA controller:

\begin{description}
    \item [reqUpdate -]
    Signal to acknowledge the received data transfer request (req) and its details, so that next request may be sent.
    Used in facilitating pop signals at the request FIFO buffer.
    \item[FLAOut -]
    32-bit FLA (Final Load Address) output sent to channels.
    Calculated by adding loadDetails with count.
    \item[FSAOut -]
    32-bit FSA (Final Store Address) output sent to channels.
    Calculated by adding storeDetails with count.
    \item[counterOut -]
    32-bit counter output sent to channels.
    \item[set0 -]
    Signal used to activate channel 0 and store FLA, FSA and counter data to it.
    \item[set1 -]
    Same as set0, but for channel 1.
    \item[interruptReq -]
    Request signal to the arbiter for sending out interrupt
    \item[interruptDetails -]
    34-bit interrupt details, sent out of the DMA top module when the controller requests the arbiter for access, and is acknowledged.
    Contains 2 bits to recognize interrupt with value ``11'', and contains ID (7 bits) of the transfer requestor.
    Bits 24--0 are currently not in use.
\end{description}

\subsubsection{Internal registers and signals}
The DMA controller also has some internal registers and signals:

\begin{description}
    \item[occupied -]
    A 2-bit register, with each bit representing a channel.
    When the DMA controller activates a channel, it also sets the corresponding bit in the occupied-register to 1.
    The occupied register is compared with the incoming active signals from the channels.
    Whenever a channel finishes and deactivates, there is a difference between the active signals and the occupied-register, and the controller compares the values and the active signals to determine which channel has finished.
    \item[requestID0 -]
    A 7-bit register, containing the ID of the requestor of the transfer currently being executed in channel 0.
    The ID is included in the interrupt signal the DMA controller sends out when the transfer is done executing in channel 0.
    \item[requestID1 -]
    Same as RequestID0, but for channel 1.
    \item[totalActive -]
    A 2-bit signal, which is the concatenation of both the active signals from the channels.
    Tp function properly, the DMA controller needs the active signals immediately when set signals are sent out, yet there is a cycle delay between sending the setX-signal until the channel is active.
    Therefore the setX-signals are also checked when totalActive is set.
    This is done by using setX OR activeX.
    \item[workDone - ]
    This is the signal used to determine whether a channel is done executing or not.
    workDone is calculated by subtracting the values of the occupied-register with the values of the totalActive signal.
    If the value is not ``00'', then a channel has finished, and the controller enters an interrupt state.
    The occupied-register is updated as well.
\end{description}

\subsection{DMA Request FIFO adapter}

The DMA controller is designed to process input data that arrives together with the request signal, but the FIFO design that is used to queue requests is not fully compatible with this setting.
The suitable signal from the FIFO buffer to the DMA controller is the empty-signal, as NOT-empty can be used as req-signal.
The reqUpdate-signal from the DMA controller can be used to pop new data.
However, the data must be popped before it can be read, causing a one cycle delay between the received req-signal and the associated data.
In addition, the first arrived data has to be popped without relying on the reqUpdate-signal, since the first NOT-empty must arrive in the controller together with the data.
And finally, when the last data is read and the buffer is empty, the controller must not pop any more content, or it will cause problems in the FIFO.
The reqUpdate signal must be ignored somehow.
   
%The reason they are not compatible can be explained by using an example where only one request has been stored into the FIFO buffer.
%When content is stored, the empty-signal from the FIFO buffer is set from '1' to '0'.
%This is the only signal that is depenendant on whether there is content or not, and should thus be used as request signal.
%However, when request is stored in the FIFO buffer, the new request is not automatically set as output.
%For this to be done, the request must be popped.
%When popped, the request data is set to the output, and the DMA controller receives it.
%But since there are no other requests stored and waiting to be processed, the FIFO buffer will count as empty, and the empty signal is set to '1' (negating any request signal into the controller).
%In summary, the problem is this: Either the controller will receive old data from a previous request with when new request, or there will be no request signal with the new data.
%This is the case when there are only 1 request.
%For a series of requests, the first request signal will be coupled with old data, while the final request will not be processed.
%A further problem is to avoid having the controller popping wrong at in the FIFO buffer.
%When a pop occurs, the FIFO buffer changes output by one \todo{sounds stupid, improve} storing cell, and it is possible to do a pop, even if the empty signal is '1'.
%The empty signal is based on whether the FIFO's top and bottom have the same index, with pop as last command, and popping moves the bottom index.
%Popping wrong will therefore create a wrong empty signal, and when processing the final request, the DMA Controller must not pop wrongly.

The challenge is to make sure that req-signals to the DMA controller arrive at the same time with the corresponding data, and to make sure that incorrect pops to not occur.
This problem is solved by adding an adapter between the request FIFO buffer and the DMA controller.
It delays the request signals from the FIFO buffer by one clock cycle, and sends out pop-signals at correct time.
Figure \ref{fig:adapter} shows how the adapter is built, and also how the adapter, the FIFO buffer and the DMA controller are all connected to each other.

\begin{figure}[htb]
    \centering
    \includegraphics[width=1.0\textwidth]{Figures/DMA/Adapter}
    \caption{Adapter between the FIFO buffer and the DMA Controller}
    \label{fig:adapter}
\end{figure}

The adapter has one register for storing request signals, but consists otherwise of combinatorics.
The NOT-empty signals are always delayed by a cycle, making sure they arrive together with the next corresponding data to the controller.
If the register's value is 0, the first NOT empty-signal will cause the adapter itself to pop the buffer, and register changes value to 1.
As long as NOT-empty is active, the adapter pops whenever reqUpdate is active.
When the buffer finally is empty, the next reqUpdate signal will simply set the register's value to 0.
%When a request is stored in the FIFO buffer, and the empty signal is sent in and negated, if the current value of the request register is 0, then the request is the first to arrive, either alone or in a series of requests.
%The register sends out the request the next cycle, which arrives in the DMA Controller together with the newly popped data.
%As long as there are more in the FIFO buffer, the negated empty signals combined with the reqUpdate-signal from the DMA controller is used to pop for new requests (NOT empty AND (REQUEST AND reqUpdate).
%ReqUpdate is set immediatly inside the DMA Controller whenever it receives a request and has a free channel.
%When a request is processed, the next request details is popped at the same, so that they may be ready for input reading.
%Using a register, any request signals are delayed with at least one cycle, so that output details from the FIFO buffer is popped and reaches the DMA Controller together with the corresponding request signal.
%There are, however, some more issues to be aware of:
%When the final data has been popped, the negated empty-signal becomes '0', but the expected case will be that while final data is ready to be read, there are no free channels to process the final request for a while.
%The request register must store the final request, without being overwritten by the empty-signal.
%This is done by using it's own value to update itself, until the next reqUpdate-signal is sent out, signaling that the DMA Controller has processed the final job.
%As long as there is no more requests, and the negated empty-signal is '0', the request register will be set to '0' as well when the reqUpdate signal is sent out.
%Notice also that as long as the negated empty-signal is '0', no pops will be made to the FIFO buffer either. 

\subsection{Manual subtractor}
One of the main components used inside both load channels and store channel is the manual subtractor.
It works by taking three inputs, two for subtraction and one as control signal.
If the control signal is active, the first input will have its value subtracted by the second input, and the output is the result.
If the control signal is not active, there will be no subtraction, and the first input value is the output value. 

\subsection{Load channel}
DMA channels are split into two parts,
a part responsible for issuing load commands, and a part responsible for issuing store commands with appropriate data.
These are called the ``load channel'' and ``store channel'', respectively, and are not to be confused with channels 0 and 1 in figure \ref{fig:TopViewFinalSimple2IO}, as they both contain these two sub-modules.
  
The load channel is responsible for generating load requests that are sent out from the DMA.
It can be seen in figure \ref{fig:loadChannel}.

\begin{figure}[h!]
    \centering
    \includegraphics[width=1.0\textwidth]{Figures/DMA/LoadChannel}
    \caption{Load channel}
    \label{fig:loadChannel}
\end{figure}

\subsubsection{Inputs}
The load channel have the following inputs:

\begin{description}
    \item[clk -]
    Input clock signal
    \item[reset -]
    Reset signal
    \item[set - ]
    Signal from DMA controller used to set all registers to the input values
    \item[LMode -]
    Value for LMode register
    \item[FLA -]
    32-bit value for the FLA register
    \item[count -]
    32-bit value for the counter register
    \item[loadAck -]
    Signal received from the arbiter, specifying that the current load request is acknowledged.
\end{description}

\subsubsection{Registers}
The load channel have the following registers:

\begin{description}
    \item[LMode -]
    Register for loading mode, connected to the subtractor.
    \item[FLA -]
    Contains Final Load address
    \item[counter -]
    Used to count down the number of load requests to be generated. Also used to calculate offset from FLA, if LMode is 1.
\end{description}

\subsubsection{Outputs}
The load channel have the following outputs:

\begin{description}
    \item[loadAdrOutput -]
    34 bit signal with the address for the current load command. Is concatenated with two bits with value ``00'', used as MSB.
    \item[loadReq -]
    Request signal from the load channel to the arbiter, for requesting output access to issue load command to the system.   
    \item[loadActive -]
    Signal sent out, used to indicate that the load channel is active.
\end{description}

With the exception of loadAck from the arbiter, all input signals are received from the DMA controller.
When set is active, all registers are overwritten by the matching input signals.
LMode determines if the load requests are for a fixed address, or a block of addresses, by activating or deactivating a manual subtractor that the FLA and the counter are connected to.
If LMode is active, the current load address is calculated by subtracting the FLA with the current counter value (minus 1).
The counter is used for both offset calculation, and counting down the number of remaining requests.
A unique number is needed to identify whether the current job is done or not, and the value 0 has been chosen for that.
As long as the counter has a value above 0, the job is not done, and both the load request and the load active signals are active.
The load channel does not need any additional data or input signals besides the registers, therefore the channel is designed to issue loads as long as the counter is not 0.
At the final count, the final load address should be issued, but if the counter value is 0, there will be no requests.
This is avoided by incrementing the Count input value by one when issuing a new job (set = 1), and decrementing the counter's current value by one before subtracting with FLA.
Thus, when issuing the final load, the counter's value is ``...001'' while the counter's output value to the manual subtractor is 0.
The load channel can issue the final load, and then the counter reaches 0 when done.
The counter is also connected to another manual subtractor, which decrements the value with 1 if activated.
Except from when the set-signal is active, the counter is always overwritten by the result from this subtractor.
This manual subtractor is activated by the loadAck signal from the arbiter.
Whenever the loadAck signal is active, the current load address output passes through the arbiter, and the counter is decremented by one.

\subsubsection{Why final loading address, and counting down the offset?}
When designing the load channel, it was needed to have at least a static address for loading, an offset from that address, and a value for counting down the remaining loads to be issued.
The needs are the same for the store channel as well.
The current design incorporates both offset and counting into one single register.
The current address can be calculated by having a starting address that is added together with the current offset.
Alternatively, one may use a final address and subtract it with the offset.
If counting down, the zero-value is distinguishable from non-zero values, and can be used to know whether the countdown is finished or not.
If a counter has an incrementing value, there must be an extra register that contains the final count value so that the load channel can know when the job is finished.
In order to utilize as few registers as possible for the task, the counter and offset value has been joined into one register, and the system counts down towards zero.
This eliminates the need for any extra final count registers, since the counter can be checked for zero or non-zero values.
This means also using the final load address and subtracting it with the offset from the counter, when calculating current address.

\subsection{Store channel}
The store channel is responsible for generating store requests, together with the correct data that has been loaded from a distinct memory address into the DMA module.
It can be seen in figure \ref{fig:storeChannel}.

\begin{figure}[htb]
    \centering
    \includegraphics[width=1.0\textwidth]{Figures/DMA/StoreChannel}
    \caption{Store Channel}
    \label{fig:storeChannel}
\end{figure}

The design of the store channel have similar inputs, registers and outputs to the store channel, which can be seen in figure \ref{fig:storeChannel}.
There are some differences of note:
\begin{description}
    \item[FSA and FLA -]
    Final store address is unique to the store channel.
    The store channel still consists of the FLA, needed to compare loaded data with the origin address.
    \item[Unique inputs and outputs -]
    Not found in the load channel.
    These are the input dataRdy, and the output loadAdrID.
    \item[storeReq and dataRdy -]
    Request signal storeReq from the store channel to the arbiter is not based on the counter, but on the input signal dataRdy.
\end{description}

Any loads issued by the associated load channel is expected to reach a data buffer shared by all channels.
All received data has an ID, informing which address in the memory the data was loaded from.
Load address ID, based on the FLA decremented with the counter, is compared to the ID of the next data in the buffer.
If there is a match, the next data belongs to this channel, and a dataRdy-signal is sent in.
In this design of the store channel, the dataRdy is used as a request signal.
The reason behind this design is to make it possible to treat the store channel as a black box from outside, so that the if the store channel is replaced or changed in the future, the outside world does not need to know anything of it.
Since there is a corresponding store address for every load address, the counter is used to calculate offsets from both the FLA and the FSA.
Notice that there is no data sent into the store channel.
The data is provided by a shared data buffer, and sent straight through to the arbiter.
The store channel sends out the store address, together with two bits for store commands.
These two bits are concatenated with the address, and are the two MSBs, with value ``01''.

\subsection{Arbiter}
The arbiter is used to arbitrate between all channels and the DMA controller, when any of them is ready to send out their output to the system.
For each clock cycle, it selects between a number of requests, and allows selected command and assosiated data to pass through.
It can be seen in figure \ref{fig:arbiter}

\begin{figure}[htb]
    \centering
    \includegraphics[width=1.0\textwidth]{Figures/DMA/Arbiter}
    \caption{Store channel}
    \label{fig:arbiter}
\end{figure}

The arbiter receives store requests and load requests from each channel, as well as interrupt requests from the DMA controller.
Each channel receives one store acknowledge signal and one load acknowledge signal from the arbiter.
The DMA controller also receives interrupt acknowledge signals from the arbiter.
For each channel, there are two 34-bit command inputs, one for storing requests and one for loading requests.
There is also a 34-bit interrupt input from the DMA controller.
All these command requests go through a 34-bit multiplexer, set by the arbiter controller, based on which request is acknowledged.
For instance, if a store request from channel 1 is selected, the store details from channel 1 is the one that passes through the multiplexer and reaches CMD out.
The data input comes from the shared data buffer, and is sent through the data multiplexer whenever there is a store request is acknowledged.
There is also a blocking signal connected to the arbiter.
If the blocking signal is active, then no requests will be acknowledged, and any work inside the DMA module, whether it is in a channel or in the DMA controller, will be halted.
The system bus is expected to be the bottleneck in this design, and any outputs from the DMA module must not be sent out faster than the system can handle.
Should there be any need, the system needs to be able to halt the DMA module from issuing any more commands until the previous commands have been handled. 

The priorities for the arbiter controller are:
\begin{itemize}
    \item Block, then interrupt, then store, then load
    \item If there are two stores or two loads competing, then the least recently used channel gets priority.
    This way the commands sent out from the DMA module will alternate between the two active channels. 
    This is separate for load requests and store requests.
\end{itemize}

\subsection{Comparator}
The comparator is used to compare the current output data's loading ID with the loading ID the requesting store channels are waiting for.
The channel that has a match receieves a dataRdy-signal from the comparator. 

\subsection{FIFO Data buffer}
The implemented data buffer is a FIFO buffer.
It is 64 bits wide, 32 bits containing the address from where the data is loaded, known as loadID, and 32 bits containing the data itself.

\subsection{Interaction in the two-channel setup}
The channels are set up in what is called the TwoChannelSetUpBuffered module.
This setup can be seen in figure \ref{fig:twoChannelSetup}.
The figure shows how two channels 0 and 1 are combined in this submodule together with the FIFO data buffer, a comparator, and an arbiter.

\begin{figure}[htb]
    \centering
    \includegraphics[width=1.1\textwidth]{Figures/DMA/TwoChannelSetUpBuffered}
    \caption{The two-channel setup, buffered version}
    \label{fig:twoChannelSetup}
\end{figure}

The reason behind this choice was to have a working environment where testing the interaction between the channels, arbiter and input buffer, were possible before having finished the DMA controller.

All inputs to the channels from the DMA controller go through synchronous buffers.
Originally, the setup was designed without buffers, but behavioural simulation showed that whenever a set-signal was received exactly at a clock edge together with altered input values, the previous counter value would be written into the counter registers in the channels.
The rest of the inputs to the channels were correctly written.
The inputs for the counters passes through some additional combinatorics before reaching the registers, so apparently these values did not reach the registers in time with the set-signal.
Adding buffers for all input signals solved this problem, by giving all inputs to the channels an entire cycle to synchronize.
The cost is an additional cycle when activating the channels.

The DMA controller, which sends in these inputs, is also connected directly to the arbiter.
Whenever the DMA controller needs to send interrupt signals, this goes through the arbiter.

Notice that each setX\_buf registers are also connected to the active-signal of their corresponding channel that are sent back to the DMA controller.
Even if buffering the inputs delays the channel activation by one cycle, the DMA controller should still receive the active-signal as early as possible.

Popping data from the data buffer at the right time is also a challenge.
Whenever a channel stores the latest data from the FIFO buffer, new data should be popped, but only if there is more.
If the buffer is empty, no more pops should be issued.
Whenever new data arrives to an empty buffer, it must be popped first before any of the channels can read the ID.
This is solved by implementing some combinatorics, and a register called popFirst.
When new data arrives, there must be a cycle between the push and the pop.
popFirst is set by combining the empty-signal with the push-signal, which are both active when the first data arrives, and popFirst will be used in popping the first data during the next cycle.
Notice that either a store-ack signal must be active, or that there must be no data-rdy signal active.
This is to avoid a potential deadlock due to data being overwritten too early.
Simulation showed that if the last data was popped, but not sent due to block or interrupt, it would be overwritten if new data arrived in the FIFO buffer before the interrupt or block was over.
Otherwise, if there is content in the buffer that is to be popped, it will be done so as long as the empty signal is low, and as long as any of the store ack-signals from the arbiter is active (meaning that data with store address is sent through and out of the DMA module).

Whenever something is sent through the system, either a store request, load request, or interrupt, an additional signal, storeOutput, is set.
This is done by combining all ack-signals with an OR gate into storeOutput.
This is to be used as a separate signal to the system that handles the output from the DMA module, to signal that a new command is available.

\subsection{Tweaks in the system}
The DMA module was designed for word addressing, while the system used to test the hashing module and the DMA module in this project uses byte-addressing.
The current design does not support both, and in order to both save time and to adapt the DMA module to the test system, the following changes were made:
\begin{itemize}
    \item The DMA controller multiplies the counter value with 4 before calculating FLA and FSA
    \item The multiplied countervalue is stored in the channels
    \item The channels decrement with 4 instead of 1 
\end{itemize}

\section{Verification and testing of DMA Module}

%\section{Functional testing}
The DMA module is tested by first testing the individual components, then their interaction with each other.

In the individual testing, the component receives different combinations of its inputs over time.
The test shows whether the component behaves as expected, based on the input.
This is done by reading the outputs based on inputs and/or values of internal registers.
Components that are tested individually are:
\begin{itemize}
    \item \textbf{Adapter} 
    \item \textbf{Arbiter Controller}
    \item \textbf{DMA Controller}
    \item \textbf{Load Channel}
    \item \textbf{Store Channel}
\end{itemize}

In the interaction testing the components are tested for their interaction with each other.
Since the components have been tested individually, they are assumed at this point to be functional. 
Internal values are thus ignored, also since the focus of the tests are to ensure that the interaction between the components are working as expected.
The interaction tests are:
\begin{description}
    \item[Arbiter -]
    Tests that internal multiplexers are set correctly by the controller, based on the selected requests.
    \item[Two Channel setup, single request -]
    First of two tests that checks if the two-channel setup works as expected, by activating a single channel.
    Data with ID is fed manually to the data buffer.
    The tests ensures that a load channel sends out number of requests according to count value, and that whenever data arrives, stores are issued and get priority above loads.
    Also tests if interrupts and blocks get highest priority, and that active signals are activated and deactivated correctly.
    Correctness is ensured by reading the output values (mainly detailsOutput and dataOutput) from the two-channel setup.
    \item[Two Channel setup, two requests -]
    Second of two tests, this one activates both channels.
    In addition to the first test, this one also ensures that loads from both channels alternates when both are active, and that correct channel identifies and stores the correct data that has been loaded.
    When one channel is done, the other channel will have full access to the arbiter, and loads will no longer alternate.
    Correctness is ensured by reading the output values.
    \item[Toplevel -]
    This is where the entire DMA module is tested, as requests are sent in.
    Each cycle with load, store, interrupt and blocking are counted (either identified by the command value, or by the blocking signal when it is high, in the test).
    For each load issued, a register receives the load address, and sends it back to the DMA module with a push signal and data (since the data value is not important for this test, each value is incremented by 8).
    Thus there is a cycle delay from a load is issued to the data arrives to the DMA module.
    In the real world, there will be far more cycles from a load is issued, until its data arrives in the DMA module.
    Whether issuing multiple loads or not is possible also depends on the system it is integrated with.
    There will be many differences between this project, where the DMA module is connected to a single shared bus network, and when the module is integrated into SHMAC, where the main network is a packet-switched network.
    In this test, however, all data arrives quickly, as if using a bus and memory that is extremely fast.
    One reason is to simplify the test by reducing the necessary waiting time in the test itself, and another is that the module is tested for near-maximum capacity.
    There are three tests in total: One with single request (only one channel active), one with two requests (both channels active) and one with four requests (test for waiting requests).
    Correctness is ensured by comparing the test counter values with expected number of loads, stores, interrupts issued, and number of blocks.
    
    It turns out that in the tests for each round of blocks, the block counter counts one block less than the number of cycles a block is active. It also turns out, when analysing the signals, that the module is inactive in the correct number of cycles as long as the block signal is active; the block counter merely does not count during the first cycle during a block. Since this is considered a flaw with the test itself and not with the DMA module, the error is considered negligable. 
\end{description}


\end{appendix}

